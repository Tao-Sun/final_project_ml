\section{Experimental Evaluation}

%This study is designed as a pilot study and serves to identify the appropriate difficulty for the Petri net material.
%The amount of places and transitions and their interconnection should make the task feasible, meaning the recall rate should be far above a randomly drawn new Petri net, but still each recalled drawing should contain some error.
%Successful analyses have been conducted with accuracy measure from 58\% correct recall \cite{egan1979chunking} up to 95\% \cite{moss2006role} which is hence taken as the desired lower and upper boundary respectively.
%
%\subsection{Method}
%
%\subsubsection{Participants}
%Three first-year students, three third-year students who have taken at least one Petri net class and three scientific employees who work with Petri net related material on a daily basis shall participate in this study.
%The students' participation in experiments is mandatory and part of their curriculum, the scientific assistants get cookies as a reward.
%
%\subsubsection{Materials}
%Petri nets are randomly generated (cf. Appendix~\ref{app:generated}) and visually formatted (cf. Appendix~\ref{app:formatted}).
%A range of 5 and 30 graph elements ($n$) is used with a number of interconnections ($f$) between $f=n-1$ and $f=round(3/2 * n)$.
%Grid lines are drawn onto the background so that each place or transition is on an intersection of one of the vertical and horizontal lines.
%
%\subsubsection{Procedure}
%In the beginning the participant is introduced in using \citetitle{renew} \cite{renew} as a tool for drawing Petri nets.
%Further participant is informed that any element which is not on one of the intersections between the vertical and horizontal lines is removed during the scoring process.
%Each participant does 13 runs from which the first run is discarded as a demo run.
%In that way each participant sees 13 different Petri nets.
%The participant sits in front of the screen and is presented one of the generated Petri nets for 5 seconds.
%After that immediately the recall phase starts.
%Using \citetitle{renew} the participant is asked to reconstruct the previously seen net.
%On the background the same grid lines are presented and the participant is asked to position the places and transitions on the grid lines as the participant has seen it before.
%After the participant believes that everything which could be remembered has been drawn, the file is saved, printed and closed.
%The participant is then asked to indicate the used strategy on the print-out and circle which symbols were grouped together.
%
%\subsection{Scoring}
%\label{subsec:scoring}
%The scoring is inspired on a previous study on mechanical engineers \cite{moss2006role} which used an extended scoring system of \textcite{chase1973mind}.
%The scoring of each recall is done in the following manner:
%
%\begin{APAenumerate}
%	\item Each place and transition which is not drawn on the grid intersections is deleted (including the arcs which connected the deleted element with other elements).
%	\item  Each place and transition which has been omitted on the grid is added to the element omission score.
%	\item Each place and transition which appears on a wrong position on the grid is added to the element insertion error score.
%	\item Each arc which connects two previously unconnected elements is added to the arc wrong connection score.
%	\item Each arc which is omitted is added to the arc omission score.  
%	\item Each arc which points into the wrong direction is added to the arc wrong direction score.
%\end{APAenumerate}
%
%The total weighted error score for a drawing is the weighted sum of all the errors mentioned above.
%Each error score is weighted with $1$ except the arc wrong direction score which is weighted with $0.5$.
%
%\subsection{Data Analysis}
%
%For each participant and each Petri net first the error score is calculated.
%Then correct recall score is calculated as a sum of all correctly positioned elements.
%Then the ratio of the correct recall score to the sum of both the total weighted error score and the correct recall score is calculated.
%If this ratio is in the range of 58 to 95\%, it is selected as appropriate material.
%Otherwise the Petri net is discarded.
%Further the Petri nets are ranked according to their ratio and the rank list is partitioned into three proportions of similar sizes.
%The partitions are labeled with their respective difficulty level, viz. easy, medium and difficult.
