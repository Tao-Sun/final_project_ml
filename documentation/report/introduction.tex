\section{Introduction}

A human brain is a complex system composed of structural regions that are functionally specialized. Due to the conclusion that these locally segregated regions are actively interconnected even when a subject is at resting-state \cite{biswal95}, the resting-state functional Magnetic Resonance Imaging (fMRI), which is a neuroimaging procedure that measures the changes of signals associated with blood flow, has become a prevailed tool for investigation of brain functional networks. Since functional connectivity in the brain is an significant measure that could indicates disease-induced changes in the network, it could provide assist to the diagnosis of brain diseases such as Alzheimer Disease (AD) or its early stage Mild Congnitive Impairment (MCI).

With the typical assumption that the functional networks in a brain is stationary, many diagnosis methods of MCI and AD with resting-state fMRI (rs-fMRI) model the network with correlation analysis such as Pearson’s correlation, independent component analysis \cite{li12}. However, recent studies \cite{hutch13} suggest that significant temporal changes exist in functional connectivity. Thus, valuable information could be lost when connectivity estimation is based on analysis restricted to a single value obtained from the entire scanning time.

In this paper, we present a novel method to classify subjects with AD and Normal healthy Control (NC) by combining Deep Auto-Encoder (DAE) and Recurrent Neural Networks (CNN). Initially rs-fMRI images data is preprocessed and mean time series of Regions of Interest (ROIs) are extracted. Then high-dimensional time-series data  is reduced to a lower dimensionality by the DAE and then splitted into multiple identical-sized sub-series. A RNN classifier is trained on the sub-series which can tag each sub-series as AD or NC. Finally, the diagnosis for a subject is made by ensemble of the outputs of the sub-series classifier. Tests shows that accuracy of the method approaches 70\% on test data.