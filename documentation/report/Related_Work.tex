\section{Related Work}

To model temporal dynamics of a brain network, \cite{eavani13} proposes a Hidden Markov Model (HMM) framework, which associates discrete hidden states with distinct connectivity patterns. \cite{suk16} builds a probabilistic model combining Deep Auto-Encoder (DAE) and HMM to model functional dynamics in rs-fMRI and estimate the likelihood of a subject as belonging to Mild Congnitive Impairment (MCI) status or NC. After preprocess, the model traines a DAE by stacking multiple RBMs for the purpose of dimension reduction. With low-dimensional data, two separate HMM models are trained for the classes of NC subjects and MCI subjects respectively. During testing, the class of the model with higher likelihood is taken as a clinical decision. These two models shares a common hypothesis, i.e. it is possible to \textquotedblleft decode connectivity dynamics into a temporal sequence of hidden network \textquoteleft states \textquoteright for each subject\textquotedblright \cite{eavani13}. However, the authors does not fully justified the hypothesis. Actually, for the diagnosis purpose, these hidden states and their definitions are not relevant.
