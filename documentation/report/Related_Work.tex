\section{Study 2}

In this study the prepared material of the last study is used in a similar setup.
As a new component a distractor task is added to deteriorate the visual-spatial sketchpad \parencites(cf.)(){baddeley1986working}.
Further a second measure, the inter-response time analysis, is taken, to gain more insight into the nature of chunks.

\subsection{Method}

\subsubsection{Participants}
10 first-year students, 10 third-year students who have taken at least one Petri net class and 10 scientific employees who work with Petri net related material on a daily basis shall participate in this study.
As number of professors and teaching assistants of the University of Hamburg might not be sufficient, assistance from other universities which are also involved in Petri net research might be asked to help.
The students' participation in experiments is mandatory and part of their curriculum, the professionals get some cookies as a reward.
Having participated in study 1 excludes the participants from study 2.

\subsubsection{Materials}

\paragraph{Petri nets} The Petri nets which have been approved in study 1.

\paragraph{Distractor task} An unsolvable 15-puzzle \parencites(cf.)(){ratner1986finding} which runs as a program on the same computer 

\subsubsection{Procedure}
In the beginning the participant is introduced in using \citetitle{renew} \cite{renew} as a tool for drawing Petri nets.
Further the participant is informed that any element which is not on one of the intersections between the vertical and horizontal lines is removed during the scoring process.
Each participant does 13 runs from which the first run is discarded as a demo run.
The pool of Petri nets for one participant contains 6 drawings connected with the distractor -- no distractor condition.
In the no-distractor condition the net will be shown immediately whereas in the distractor condition between the presentation and the recall a 30s delay happens.
The six drawings shown in both conditions equal 12 runs.

For each run the Petri net and its connected condition is drawn randomly from the pool.
The participant sits in front of the screen and is presented the drawn Petri net for 5 seconds.
Depending on the drawn condition either the distractor task is shown or the recall starts immediately.
Using \citetitle{renew} the participant is asked to reconstruct the previously seen net.
Mouse clicks and the screen are recorded throughout the process.
On the background the same grid lines are presented and the participant is asked to position the places and transitions on the grid lines as the participant has seen it before.
After the participant believes that everything which could be remembered has been drawn, the file is saved, printed and closed.
The participant is then asked to indicate the used strategy on the print-out and circle which symbols were grouped together.

\subsection{Data Analysis}

\subsubsection{Performance Analysis}

First the error scores are calculated as described in \textit{scoring} of study 1.
A mixed-design analysis of variance (ANOVA) with 
the Petri net difficulty level (simple, medium, difficult),
the distractor task and
the total weighted error score
as a within-subjects factor and 
expertise level
as the between-subjects factor is conducted.
As a post-hoc test Tukey's HSD is chosen.
Significant differences are expected between the three groups and between the three Petri net difficulty levels.
First-year students are expected to show significant differences between the distractor task and the no distractor task condition while scientific employees are expected not to have a significant difference.

\subsubsection{Inter-Response Time Analysis}
The mouse clicks and screen recordings are analyzed using a single-linkage hierarchical clustering algorithm \parencites(cf.)(){moss2006role}.
The found chunks are supposed to correspond to the units indicated by the respective participant.
Since each drawing was both presented in the distractor and no distractor condition, for the same drawing for each participant two hierarchical clusters exist.
They are compared using the correlation between the cophenetics of each hierarchical cluster \parencites(cf.)(){fowlkes1983method}.
The two cophenetics are expected to strongly correlate for scientific employees since their are expected to be unaffected by the distractor condition whereas a rather low correlation between the two cophenetics is expected for first-year students. 
In the no distractor condition first-year students are expected to have rather complex (probably both nested and overlapping) chunks in working memory which help to recall the drawing with comparably little error.
In the distractor condition first-year students need to rely on incidental encoding to their long-term working memory.
Maybe the presented drawing can utilize existing knowledge about superficially similar visual languages like flow chart diagrams or other, probably highly personal, mnemonic strategies.
Without making elaborated assumptions about the used strategies, they are supposed to result in a different cluster than in the immediate recall condition.
